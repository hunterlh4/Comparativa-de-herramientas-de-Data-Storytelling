
\documentclass[preprint,12pt]{elsarticle}
\usepackage[utf8]{inputenc}
\usepackage[spanish]{babel}
\usepackage{amsmath}
\usepackage{amsfonts}
\usepackage{amssymb}
\usepackage{graphicx}
\usepackage{url}
\usepackage{natbib}


\begin{document}
	
	\begin{frontmatter}

		\title{\huge  Comparativa de herramientas de Data Storytelling }
		\author{Tarqui Montalico,Risther(2017057469)}
		\author{Liendo Velásquez, Joaquin (2016054463)}
		\author{Limache Victorio, Piero (2017057857)}
		\author{Ticona Mamani, Alex (2017057860)}
		\author{Sánchez Rodriguez Bayron Andres (2017057859) }
		\author{Davi Quintanilla Fernández (0099019860)}
		\address{Tacna, Perú}
		


%%INICIO abstract
\begin{abstract}
This tool is a structured approach to how we communicate insights from data. For organizations it is very important to include stories in the analyzes, because it helps to easily understand and persuade to generate a change in people's actions. The role of data translator allows making data-based discoveries understandable, as well as applying different journalistic strategies. Developing communication, listening and speaking skills is important for Data Storytelling to be a guaranteed success.
\end{abstract}
%%FIN abstract


\end{frontmatter}

%%INICIO Resumen
\section{Resumen}
Esta herramienta consiste en un enfoque estructurado sobre cómo comunicamos insights a partir de los datos. Para las organizaciones es muy importante incluir historias en los análisis, debido a que ayuda a un fácil entendimiento y a persuadir para generar un cambio en las acciones de las personas. El rol de traductor de datos permite hacer entendible los descubrimientos basados en datos, así como también aplicar distintas estrategias periodísticas. Desarrollar habilidades en comunicación, en escucha y oratoria es importante para que el Data Storytelling tenga un éxito asegurado.
%%FIN Resumen


%%INICIO Introducción
\section{Introducción}
En la escuela, aprendemos mucho sobre lenguaje y matemáticas. Por el lado del lenguaje, aprendemos cómo unir palabras en oraciones e historias. Con las matemáticas aprendemos a entender los números. Pero es raro que estos dos lados estén emparejados: nadie nos enseña cómo contar historias con números. Además del desafío, muy pocas personas se sienten naturalmente adeptas a este espacio.
\\
\\
Esto nos deja mal preparados para una tarea importante y cada vez más demandada. La tecnología nos ha permitido acumular cantidades cada vez mayores de datos y existe un deseo creciente de acompañamiento de dar sentido a todos estos datos. Poder visualizar datos y contar historias con ellos es clave para convertirlos en información que se puede utilizar para impulsar una mejor toma de decisiones.
En ausencia de habilidades naturales o capacitación en este espacio, a menudo terminamos confiando en nuestras herramientas para comprender las mejores prácticas. Avances en la tecnología, además de incrementar la cantidad y el acceso a
datos, también han creado herramientas para trabajar con datos omnipresentes. Prácticamente cualquiera puede poner algunos datos en una aplicación de gráficos (por ejemplo, Excel) y crear un gráfico. Es importante considerar esto, así que lo repetiré: cualquiera puede poner algunos datos en una aplicación de gráficos y crear un gráfico. Esto es notable, considerando que el proceso de creación de un gráfico fue históricamente reservado para los científicos o aquellos en otros roles altamente técnicos. Y aterrador, porque sin un camino claro a seguir, nuestras mejores intenciones y esfuerzos (combinados con los valores predeterminados de herramientas a menudo cuestionables) pueden llevarnos en algunas direcciones realmente malas: 3D, colores sin sentido, gráficos circulares.	\\

%%FIN Introducción


%%INICIO Marco Teórico
\section{Desarrollo}

%%----------------------------------------------------------------------------------------------------------------------------------------------------------
	\subsection{\textbf{¿Qué es Data StoryTelling?}}
	La narración de datos o el Data Storytelling es el proceso de traducir los análisis de datos a términos simples para influir en una decisión o acción comercial. Este término se ha asociado con diferentes funciones: visualizaciones de datos, infografías, tableros, etc. El Data Storytelling es mucho más que eso, son formas de comunicar información con datos, imágenes y narrativa. Con el crecimiento de los negocios digitales, la toma de decisiones basada en los datos se ha convertido en algo primordial a menudo asociado con la analítica y la ciencia de los datos.
\\
\\
Para comprender mejor qué es el Data Storytelling, hay que centrarse en estos tres elementos (dato, imagen y narrativa) y ver cómo trabajan juntos. Lo que se busca es dar voz a los datos y comunicar los resultados del análisis con narrativas. Es hacer de una presentación aburrida una historia con datos entretenida. Aquello que no llamaba la atención sin elementos visuales como son los  gráficos o las tablas, ahora cobra forma bajo el Storytelling.
\\
\\

\begin{figure}[htb]
		\begin{center}
			\includegraphics[width=12cm]{img/img1.png} 
		\end{center}
	\end{figure}

	\subsection{\textbf{Elementos}}



La conjunción de narrativa y datos permite explicar por qué un insight puede ser tan importante. A su vez, cuando añadimos una visualización a nuestros datos, podemos iluminar (“enlighten”) a la audiencia con insights que no habían visto de otra manera. Por otro lado, la combinación entre narrativa y visualización genera engagement o interés en la audiencia. Si unimos narrativa, datos y visualización logramos influenciar y generar un impacto en las acciones de las personas.


%%****

\subsection{\textbf{Ventajas}}

	\begin{itemize}
	\item Identificar y actuar rápido sobre tendencias emergentes: incluso los archivos de datos casi infinitos empiezan a tener sentido al representarse gráficamente; lo que nos permite detectar parámetros que están altamente correlacionados.
	
	\item Comprensión ágil de la información: las representaciones gráficas permiten ver grandes cantidades de datos de forma clara y coherente, lo que facilita la extracción de conclusiones e insights
	 
	\item Crear un nuevo lenguaje de negocio para contar la historia a otros: una vez que hemos descubierto nuevos insights, el siguiente paso es comunicarlos a través de gráficos simples o visualizaciones elaboradas para lograr engagement. 
	
	\item Encontrar relaciones y patrones dentro de los activos digitales: descubrir tendencias dentro de los datos nos puede dar una ventaja competitiva, como detectar puntos clave que están afectando a la calidad del producto o solucionar problemas antes que se vuelvan más complicados.
	
	
	
	\end{itemize}


%%****
	\subsection{\textbf{Comparativa de herramientas}}
	\resizebox{14cm}{!} {
\begin{tabular}{| c | c |c |c |c |c |c |}
\hline
""& Qlik Sense & Sisense & Tableau & Domo & Dundas & Power BI \\ \hline
Eficiencia & SI	& SI &	SI &	N/A &	SI &	SI\\

Presentación de los datos & 
Conocimiento técnico (QlikView) &
SI	& SI &	SI &	SI&	SI \\

Facilidad de uso & 
Conocimiento técnico &
SI	& SI &	SI &	Conocimiento técnico &	SI \\

OLAP & 
SI &
SI	& SI &	SI &	SI&	SI \\

Visualización interactiva & 
SI &
SI	& SI &	SI &	SI&	SI \\

Conexión BD & 
SI &
SI	& SI &	SI &	SI&	SI \\

Rendimiento & 
Optimo & Optimo &Optimo &Optimo &Optimo &Optimo  \\

Multiplataforma & 
Windows &
Windows	& Windows Mac &	Full &	Windows&	Windows \\

Seguridad & 
Alta  &
Alta	& Alta &	Regular &	Alta &	Regular \\

Perfiles de usuarios & 
SI  & SI	& SI &	SI &	SI&	SI \\

ETL & 
SI &
SI	& SI &	SI &	SI&	SI \\

Modelos dimensionales & 
SI &
SI	& SI &	SI &	SI&	SI \\

Soporte en línea & 
SI &
SI	& SI &	SI &	SI&	SI \\

Licenciamiento & 
Comercial &
Comercial	& Comercial &	Comercial &	Comercial&	Comercial \\

Manejo de volumen de datos & 
SI &
SI	& SI &	SI &	SI&	SI \\

Método de distribución & 
Local	 &Local &	Local	&SAAS	&Local	&Local \\

Disco Duro & 
3 GB	&4 GB	&15 GB&	N/A	&50 GB	&1 GB \\

Memoria & 
8 GB	&8 GB	&8 GB	&N/A&	8 GB	&1 GB \\

Hardware & 
SI	&SI	&SI	&NO&	SI	&SI \\



 \hline

\end{tabular}}

	


	\section{\textbf{Conclusiones}}
	
	Cada una de las herramientas con ayuda del monitor de recursos de Windows al momento de realizar operaciones en las mismas y se encontró que los recursos mantuvieron un estado óptimo durante su ejecución. Domo al ser una herramienta netamente Cloud ejecuta todas las operaciones en sus servidores y solo requiere de un navegador web para su uso.\\
	
Los resultados arrojados en los diferentes dashboard diseñados en las herramientas fueron fáciles de comprender y no requieren de conocimientos técnicos por parte del usuario para su entendimiento exceptuando el caso de uno de los productos de Qlik (QlikView), el cuál presentó complejidad al momento de realizar operaciones en los gráficos, pues utiliza un lenguaje nativo del producto. La información desplegada en los dashboards se presenta como un gran apoyo para la toma de decisiones a nivel operacional o estratégico.\\
	
Entre las diferentes herramientas y el usuario se obtuvo un grado alto de satisfacción en cuanto a la facilidad con la que se interactúa con los gráficos, brindando así libertad de explorar, buscar y pivotar entre los diferentes objetos. Domo y Tableau sólo permiten manejar un gráfico por cada hoja; A diferencia de las demás herramientas, donde se permite trabajar diversos gráficos en una sola hoja.


	

%%****
	\section{\textbf{Recomendaciones}}
	El análisis propuesto en la investigación se basó en las herramientas de inteligencia de negocio que actualmente son tendencia, pero existen varias opciones que pueden ajustarse a los requerimientos del área, estas podrían ser una alternativa a las propuestas en el presente trabajo y se abre la posibilidad de realizar un estudio posterior con ellas.\\
 
Por otro lado, una recomendación pertinente se da en la prospección de los clientes nuevos en el área, debido a que se necesita un estudio de las acciones que tiene dado cliente con su capital, de acuerdo a las entrevistas realizadas al área, y actualmente este proceso lleva tiempo debido a que la información se encuentra en diferentes fuentes de datos, por ende el uso de esta tecnología permitiría correlacionar estos datos con respecto a unos criterios que maneja el área siendo un beneficio y evidenciando los posibles clientes a los cuales acercarse, y en el momento de una visita tener la información necesaria para poder mostrarle al cliente que beneficios podría tener al entablar relaciones con el banco o el área como tal.

%%****
\newpage
	\section{\textbf{Bibliografia}}
http://polux.unipiloto.edu.co:8080/00004523.pdf \\

https://www.ionos.es/digitalguide/online-marketing/vender-en-internet/herramientas-de-storytelling-a-examen/  \\

https://www.cyberclick.es/numerical-blog/data-storytelling-que-es-y-como-contar-buenas-historias-con-datos 



%%****
	
%%-----------------------------------------------------------------------------
	





\end{document}
